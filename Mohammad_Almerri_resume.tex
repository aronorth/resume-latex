\documentclass{resume} 
\usepackage[left=0.25in,top=0.25in,right=0.25in,bottom=0.5in]{geometry} 
\usepackage{hyperref}
\hypersetup{
    colorlinks=true,
    urlcolor=blue
}
\usepackage{tabularx}
\newcommand{\tab}[1]{\hspace{.2667\textwidth}\rlap{#1}}
\newcommand{\itab}[2]{\hspace{0em}\rlap{#1}}


\name{Mohammad Almerri} % Your name
\address{(+1)202-568-5475 \\ malmerri42@gmail.com}
\address{Indianapolis \\ IN 46202}
\address{Github: \href{https://github.com/malmerri42}{malmerri42}}

\begin{document}
\begin{rSection}{Education}

{\bf Purdue School of Engineering and Technology, IUPUI} \hfill  January 2019 - December 2021
\\ MS in Electrical and Computer Engineering with Thesis.\hfill { GPA: 3.9/4.0 }

{\bf Purdue School of Engineering and Technology, IUPUI} \hfill  August 2013 - May 2018 
\\ BS in Computer Engineering with a Minor in Mathematics.
\hfill { GPA: 3.8/4.0 }


\end{rSection}

\begin{rSection}{Technical Skills}
\begin{tabularx}{\textwidth}{ @{} >{\bfseries}l @{\hspace{6ex}}X }
Languages: & C++, Python, matlab, Go-Lang, C, Embedded C, Assembly-ARM, Java  \\
Tools and Technology: & Linux (Ubuntu, Red Hat, Raspbian), Latex, RTOS, NS3, TensorFlow, Keras, Robot Operating System (ROS), Hadoop, Amazon-AWS\\
Databases: & MySql, MongoDB \\
Version Control: & Git, Github
\end{tabularx}
\end{rSection}

\begin{rSection}{Experience}

\begin{rSubsection}{Senior PK/PD Scientist}{July 2022 - Current}{{Eli Lilly and Company}}{}
    \item Analyzed clinical trial data (phase 1 to 3) to support the development of medical compounds.
    \item Exploratory data analysis, non-compartmental analysis, and population modeling. 
    \item Automating analysis workflows.
\end{rSubsection}
\begin{rSubsection}{Volunteer Researcher}{Feb 2022 - July 2022}{{Purdue Research Faculty: Zina Ben Miled}}{}
    \item Pursuing AI research, and assisting in the research of other students in the same laboratory.
    \item Built a novel AI model to translate black-box models.
\end{rSubsection}

\end{rSection}

\begin{rSection}{Publications}

\begin{rSubsection}{Global Translation of Classification Models}{May 2022}{{Information@MDPI}}{}
    \item A peer-reviewed journal paper on translating A.I models with an emphasis on explainability.
\end{rSubsection}
\begin{rSubsection}{Global Translation of Machine Learning Models to Interpretable Models}{December 2021}{{Master's Thesis}}{}
    \item A Thesis on the development of a input-agnostic globally interpretable model to model translation in artificial intelligence.
\end{rSubsection}

\end{rSection}


% 
% 
% \newpage
\begin{rSection}{Projects}

\begin{rSubsection}{MTRE-PAN}{Spring 2020 - Fall 2021}{{Master's Thesis}}{}
    \item Developed a novel interpretable hybrid model for neural networks using decision trees and SVM.
    \item Creates an interpretable clone of a trained machine learning model to facilitate auditing.
    \item Programmed in C++, using MLPACK and Armadillo libraries.
\end{rSubsection}

%   \begin{rSubsection}{Petri-Net Controller for Cat and Mouse problem}{Fall 2020}{\em Discrete event dynamic systems}{}
%       \item Built a Petri-Net Controller in Matlab, controlling doors in a home to maximize movement while preventing catastrophic failure in the Cat and Mouse problem.
%   \end{rSubsection}

\begin{rSubsection}{Pun Generator}{Spring 2019}{{Database Management Systems}}{}
    \item Deployed a webpage based pun generator hosted on an Amazon-EC2 Linux server.
    \item Featuring multi-level user registration and permission front-end, developed with Go-Lang, python and HTML. Server back-end developed using both relational and non-relational Databases (mysql, MongoDB).
\end{rSubsection}

\begin{rSubsection}{AgBot Weed and Feed competition \emph{2nd place (university all time high)}}{Spring 2018 - Fall 2019}{{Senior Design}}{}
    \item Converted a Yamaha 4x4 into an autonomous robot capable of fertilizing crops and recognizing one of three native Indiana crop weeds via image classification, and then spraying it with a specific weedkiller.
    \item Extensive multi-team, multi-shareholder, multi-department, multi-sponsor project.
    \item Ground up conversion combining mechanical, electrical and computer engineering.
\end{rSubsection}

\begin{rSubsection}{Predicting Power Outages Using NOAA Weather Data and ANN}{Fall 2017}{{Artificial Intelligence}}{}
    \item Programmed from scratch an artificial neural network to predict weather-related power outages in northern Indiana using C++.
\end{rSubsection}

\end{rSection}

\end{document}
