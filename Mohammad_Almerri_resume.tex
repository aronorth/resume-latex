\documentclass{resume} 
\usepackage[left=0.25in,top=0.25in,right=0.25in,bottom=0.5in]{geometry} 
\usepackage{hyperref}
\hypersetup{
    colorlinks=true,
    urlcolor=blue
}
\usepackage{tabularx}
\newcommand{\tab}[1]{\hspace{.2667\textwidth}\rlap{#1}}
\newcommand{\itab}[2]{\hspace{0em}\rlap{#1}}


\name{Mohammad Almerri} % Your name
\address{(+1)202-568-5475 \\ malmerri42@gmail.com}
\address{Indianapolis \\ IN 46202}
\address{Github: \href{https://github.com/malmerri42}{malmerri42}}

\begin{document}
\begin{rSection}{Education}

{\bf Purdue School of Engineering and Technology, IUPUI} \hfill  January 2019 - December 2021
\\ MS in Electrical and Computer Engineering with Thesis.\hfill { GPA: 3.9/4.0 }

{\bf Purdue School of Engineering and Technology, IUPUI} \hfill  August 2013 - May 2018 
\\ BS in Computer Engineering with a Minor in Mathematics.
\hfill { GPA: 3.8/4.0 }


\end{rSection}

\begin{rSection}{Technical Skills}
\begin{tabularx}{\textwidth}{ @{} >{\bfseries}l @{\hspace{6ex}}X }
Languages: & C++, Python, matlab, Go-Lang, C, Embedded C, Assembly-ARM, Java  \\
Tools and Technology: & Linux (Ubuntu, Red Hat, Raspbian), Latex, RTOS, NS3, TensorFlow, Keras, Robot Operating System (ROS), Amazon-AWS\\
Databases: & MySql, MongoDB \\
Version Control: & Git, Github
\end{tabularx}
\end{rSection}

\begin{rSection}{Publication}

\begin{rSubsection}{Global Translation of Machine Learning Models to Interpretable Models}{December 2021}{{Master's Thesis}}{}
    \item A Thesis on the development of a input-agnostic globally interpretable model to model translation in artificial intelligence.
\end{rSubsection}

\end{rSection}


% 
% 
% \newpage
\begin{rSection}{Projects}

\begin{rSubsection}{MTRE-PAN}{Spring 2020 - Fall 2021}{{Master's Thesis}}{}
    \item Developed a novel interpretable hybrid model for neural networks using decision trees and SVM.
    \item Creates an interpretable clone of a trained machine learning model to facilitate auditing.
    \item Programmed in C++, using MLPACK and Armadillo libraries.
\end{rSubsection}

%   \begin{rSubsection}{Petri-Net Controller for Cat and Mouse problem}{Fall 2020}{\em Discrete event dynamic systems}{}
%       \item Built a Petri-Net Controller in Matlab, controlling doors in a home to maximize movement while preventing catastrophic failure in the Cat and Mouse problem.
%   \end{rSubsection}

\begin{rSubsection}{Data Classification}{Spring 2020}{{Optimization methods for systems and controls}}{}
    \item Developed both parametric (multilayer) and non-parametric (KNN, Parzen Window) classifiers, and the techniques required to train them.
\end{rSubsection}

\begin{rSubsection}{US Grant-maker similarity as a function of proximity}{Fall 2019}{{Social Networks with machine learning}}{}
    \item Utilized single and multilayered classifiers to predict labels of clustered data as part of multidisciplinary project.
\end{rSubsection}

\begin{rSubsection}{Pun Generator}{Spring 2019}{{Database Management Systems}}{}
    \item Deployed a webpage based pun generator hosted on an Amazon-EC2 Linux server.
    \item Featuring multi-level user registration and permission front-end, developed with Go-Lang, python and HTML. Server back-end developed using both relational and non-relational Databases (mysql, MongoDB).
\end{rSubsection}

\begin{rSubsection}{AgBot Weed and Feed competition \emph{2nd place (university all time high)}}{Spring 2018 - Fall 2019}{{Senior Design}}{}
    \item Converted a Yamaha 4x4 into an autonomous robot capable of fertilizing crops and recognizing one of three native Indiana crop weeds via image classification, and then spraying it with a specific weedkiller.
    \item Extensive multi-team, multi-shareholder, multi-department, multi-sponsor project.
    \item Ground up conversion combining mechanical, electrical and computer engineering.
\end{rSubsection}

\begin{rSubsection}{ESP32 CAN-Bus Module for autonomous vehicles}{Spring 2018}{{Embedded Systems}}{}
    \item Designed and programmed with Embedded C a fully featured multithreaded ESP32 microcontroller based circuit to assist in autonomous vehicle research.
    \item Featuring CAN-Bus reader, SDCard storage, Internet connectivity, GPS and a Camera.
    \item Alternative project personally requested by faculty.
\end{rSubsection}

%   \begin{rSubsection}{ESP32 Sensor Anaylsis Module}{Spring 2018}{\em Operating Systems and Systems Programming}{}
%       \item Built and programmed with Embedded C, an ESP32 Module designed to test the specifications of a distance sensor.
%       \item Multithreaded, uses WiFi to access data and four sensors running on multithreaded processes.
%   \end{rSubsection}

\begin{rSubsection}{Predicting Power Outages Using NOAA Weather Data and ANN}{Fall 2017}{{Artificial Intelligence}}{}
    \item Programmed from scratch an artificial neural network to predict weather-related power outages in northern Indiana using C++.
\end{rSubsection}

\end{rSection}
% 
\begin{rSection}{Academic Achievements} 
    \item Honors college Undergraduate member \hfill Fall 2016
    %\item Multiple-time top 100 student nominee.
    \item Tau Beta Pi engineering honor society member \hfill Fall 2016
    \item Deans list \hfill Fall 2013 - Spring 2018
\end{rSection}

\begin{rSection}{Extra-Curricular} 
    \item Organized, developed and taught a AI-focused week long program to high school students as part of an outreach program to encourage women in STEM. \hfill Summer 2020
%\item Extensively privately tutored other students for free, as a result faculty have requested me to act as a Teaching Assistant on multiple occasions. \hfill \emph{Nearly entire school career}
\end{rSection}

\end{document}
