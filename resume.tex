\documentclass{resume} 
\usepackage[left=0.75in,top=0.3in,right=0.75in,bottom=0.6in]{geometry} 
\usepackage{hyperref}
\usepackage{tabularx}
\newcommand{\tab}[1]{\hspace{.2667\textwidth}\rlap{#1}}
\newcommand{\itab}[2]{\hspace{0em}\rlap{#1}}
\name{Mohammad Almerri} % Your name
\address{(+1)202-568-5475 \\ malmerri42@gmail.com}
\address{Indianapolis \\ IN 46202}

\begin{document}

\begin{rSection}{Education}

{\bf Purdue School of Engineering and Technology, IUPUI} \hfill {\em January 2019 - December 2021} 
\\ Thesis based MS in Computer Engineering.\hfill { GPA: 3.9 }

{\bf Purdue School of Engineering and Technology, IUPUI} \hfill {\em August 2013 - May 2018} 
\\ BS in Computer Engineering with a Minor in Mathematics.
\hfill { GPA: 3.8 }


\end{rSection}

\begin{rSection}{Technical Skills}
\begin{tabularx}{\textwidth}{ @{} >{\bfseries}l @{\hspace{6ex}}X }
Languages: & Assembly-ARM, Embedded C, C, C++, Go-Lang, Python, Java, matlab  \\

Tools and Technology: & Linux (Ubuntu, Red Hat, Raspbian), RTOSLatex, Git, NS3, TensorFlow: Keras, Robot Operating System (ROS), Amazon-AWS\\
Databases: & MySql, MongoDB \\
Version Control: & Github
\end{tabularx}
\end{rSection}

\begin{rSection}{Publication}
    mTRE-PAN: A Thesis on the development of a input-agnostic globally interpretable model to model translation in artificial intelligence. \hfill {\em Expected publication: December 2021}
\end{rSection}

%   \begin{rSection}{Projects}
%       {\bf A study on US Non-Profit Grant-maker similarity as a function of proximity}, {\em Social Networks with machine learning} \hfill {\em Fall 2019}
%       \\A collaborative project between the school of philanthropy and ECE. Time variant donation data was used to partition US grantmakers and the nonprofits they served into a bipartite graph. An RNN and a single layer classifier were trained in the data and used to develop a novel similarity index (Jdist) [\href{https://www.youtube.com/watch?v=kdroF47w-b8}{Link}].

%   {\bf GitHub Notifier}
%   \\This project aims at providing real time information of events from GitHub and notify you accordingly. The project is in ready-to-deployment stage on a demo server as a cron-job. The notification engine  used for real time tracking is completely based on the python implementation assembled in an Android App with firebase cloud support.

%   \end{rSection}



% 
% 
% \newpage
\begin{rSection}{Projects}
\begin{rSubsection}{mTRE-PAN}{Spring 2020 - Fall 2021}{\em Master's Thesis}{}
 \item A novel ensemble model designed to tackle the problem of "black box" non-accountability in AI.
 \item capable of extension to any existing pre-trained/deployed model without any interruption.
\end{rSubsection}
\begin{rSubsection}{Petri-Net Controller for Cat and Mouse problem}{Fall 2020}{\em Discrete event dynamic systems}{}
 \item Built a Petri-Net Controller in Matlab, controlling doors in a home to maximize movement while preventing catastrophic failure in the Cat and Mouse problem.
\end{rSubsection}
\begin{rSubsection}{Data Classification}{Spring 2020}{\em Optimization methods for systems and controls}{}
 \item Developed both parametric (multilayer) and non-parametric (KNN, Parzen Window) classifiers, and the techniques required to train them.
\end{rSubsection}
\begin{rSubsection}{US Grant-maker similarity as a function of proximity}{Fall 2019}{\em Social Networks with machine learning}{}
 \item Utilized single and multilayered classifiers To predict labels of clustered data as part of multidisciplinary project.
\end{rSubsection}
\begin{rSubsection}{Pun Generator}{Spring 2019}{\em Database Management Systems}{}
 \item Deployed a webpage based pun generator hosted on an Amazon-EC2 Linux server.
 \item Featuring multi-level user registration and permission front-end, developed with Go-Lang, python and HTML. Server back-end developed used both relational and non-relational Databases (mysql, MongoDB).
\end{rSubsection}
\begin{rSubsection}{AgBot Weed and Feed competition \em{2nd place (university all time high)}}{Spring 2018 - Fall 2019}{\em Senior Design}{}
 \item Converted a Yamaha 4x4 into an autonomous robot capable of fertilizing crops and recognizing one of three native Indiana crop weeds via image classification, and then spraying it with a specific weedkiller.
 \item Extensive multi-team, multi-shareholder, multi-department, multi-sponsor project.
 \item Ground up conversion combining mechanical, electrical and computer engineering.
\end{rSubsection}
\begin{rSubsection}{Predicting Power Outages Using NOAA Weather and ANN}{Fall 2017}{\em Artificial Intelligence}{}
 \item Programmed from scratch an artificial neural network to predict weather caused power outages in northern Indiana using C++.
\end{rSubsection}
\end{rSection}
% 
\begin{rSection}{Academic Achievements} 
\item Honors college Undergraduate member.
\item Tau Beta Pi engineering honor society member.
\item Deans list member from Fall 2015 onward.

\end{rSection}

\begin{rSection}{Extra-Curricular} 
\item As part of an outreach program to encourage women in STEM: organized, developed and taught a AI-focused week long program to high school students. \hfill \em{Summer 2019}
\item Extensively privately tutored other students for free, as a result faculty have requested me to act as a Teaching Assistant on multiple occasions. \hfill \em{Nearly entire school career}
\end{rSection}

\end{document}
